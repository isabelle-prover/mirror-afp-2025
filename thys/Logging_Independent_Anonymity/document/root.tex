\documentclass[11pt,a4paper,fleqn]{article}
\usepackage[T1]{fontenc}
\usepackage{isabelle,isabellesym}
\usepackage{tabto}

% further packages required for unusual symbols (see also
% isabellesym.sty), use only when needed

%\usepackage{amssymb}
  %for \<leadsto>, \<box>, \<diamond>, \<sqsupset>, \<mho>, \<Join>,
  %\<lhd>, \<lesssim>, \<greatersim>, \<lessapprox>, \<greaterapprox>,
  %\<triangleq>, \<yen>, \<lozenge>

%\usepackage{eurosym}
  %for \<euro>

%\usepackage[only,bigsqcap]{stmaryrd}
  %for \<Sqinter>

%\usepackage{eufrak}
  %for \<AA> ... \<ZZ>, \<aa> ... \<zz> (also included in amssymb)

%\usepackage{textcomp}
  %for \<onequarter>, \<onehalf>, \<threequarters>, \<degree>, \<cent>,
  %\<currency>

% this should be the last package used
\usepackage{pdfsetup}

% urls in roman style, theory text in math-similar italics
\urlstyle{rm}
\isabellestyle{it}

% for uniform font size
%\renewcommand{\isastyle}{\isastyleminor}


\begin{document}

\title{Logging-independent Message Anonymity\\in the Relational Method}
\author{Pasquale Noce\\Software Engineer at HID Global, Italy\\pasquale dot noce dot lavoro at gmail dot com\\pasquale dot noce at hidglobal dot com}
\maketitle

\begin{abstract}
In the context of formal cryptographic protocol verification, logging-independent message anonymity
is the property for a given message to remain anonymous despite the attacker's capability of mapping
messages of that sort to agents based on some intrinsic feature of such messages, rather than by
logging the messages exchanged by legitimate agents as with logging-dependent message anonymity.

This paper illustrates how logging-independent message anonymity can be formalized according to the
relational method for formal protocol verification by considering a real-world protocol, namely the
Restricted Identification one by the BSI. This sample model is used to verify that the pseudonymous
identifiers output by user identification tokens remain anonymous under the expected conditions.
\end{abstract}

\tableofcontents

% sane default for proof documents
\parindent 0pt\parskip 0.5ex

% generated text of all theories
\input{session}

% bibliography
\bibliographystyle{abbrv}
\bibliography{root}

\end{document}
