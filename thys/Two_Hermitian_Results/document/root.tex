\documentclass[11pt,a4paper]{article}
\usepackage[T1]{fontenc}
\usepackage{isabelle,isabellesym}
\usepackage{eufrak}

% this should be the last package used
\usepackage{pdfsetup}

% urls in roman style, theory text in math-similar italics
\urlstyle{rm}
\isabellestyle{it}


\begin{document}

\title{Two Theorems on Hermitian Matrices}
\author{Sage Binder and Zilin Jiang}
\maketitle

\begin{abstract}
    We formalize two results on Hermitian matrices.
    First, Sylvester's criterion: a Hermitian matrix
    is positive definite if and only if all its leading principal submatrices
    have positive determinant.
    Second, Cauchy's eigenvalue interlacing theorem:
    given a principal submatrix $B$ of a Hermitian matrix $A$,
    the eigenvalues of $B$ interlace those of $A$.

    Our approach to Sylvester's criterion is fairly standard,
    and required us to formalize Schur's block matrix determinant formula,
    which gives a formula for the determinant of a block matrix $(A, B, C, D)$
    when $A$ is invertible.

    Our approach to Cauchy's eigenvalue interlacing theorem
    follows a proof given in a set of lecture notes by Dr. David Bindel \cite{bindel2019lecture}.
    This approach involved formalizing the Courant-Fischer minimax theorem
    (a theorem about the Rayleigh quotient, which we define in this entry).
    In our statement of the Courant-Fischer minimax theorem,
    we refer to the infimum and supremum instead of the minimum and maximum,
    as this simplifies the proof and is sufficient to prove Cauchy's eigenvalue interlacing theorem.
\end{abstract}

\tableofcontents

% include generated text of all theories
\input{session}

\bibliographystyle{abbrv}
\bibliography{root}

\end{document}
