\documentclass[11pt,a4paper]{article}
\usepackage[T1]{fontenc}
\usepackage{isabelle,isabellesym}

\usepackage{url}
\usepackage{amssymb}
\usepackage{xspace}

% this should be the last package used
\usepackage{pdfsetup}

% urls in roman style, theory text in math-similar italics
\urlstyle{rm}
\isabellestyle{it}

\newcommand\isafor{\textsf{Isa\kern-0.15exF\kern-0.15exo\kern-0.15exR}}
\newcommand\ceta{\textsf{C\kern-0.15exe\kern-0.45exT\kern-0.45exA}}

\begin{document}

\title{A Formalization of Weighted Path Orders and Recursive Path Orders\footnote{Supported by FWF (Austrian Science Fund) projects P27502 and Y757.}}
\author{Christian Sternagel \and Ren\'e Thiemann \and Akihisa Yamada}
\maketitle

\begin{abstract}
We define the weighted path order (WPO) and formalize several
properties such as strong normalization, the subterm property, and closure properties
under substitutions and contexts. Our definition of WPO extends the original definition by also permitting
multiset comparisons of arguments instead of just lexicographic extensions. Therefore, our WPO not only
subsumes lexicographic path orders (LPO), but also recursive path orders (RPO).
We formally prove these subsumptions and therefore all of the mentioned properties of WPO
are automatically transferable to LPO and RPO as well. 
Such a transformation is not required for Knuth--Bendix orders (KBO),
since they have already been formalized. Nevertheless, we still provide a proof that WPO subsumes KBO
and thereby underline the generality of WPO. 
\end{abstract}

\tableofcontents

\section{Introduction}

Path orders are well-founded orders on terms that are useful for automated deduction,
e.g., for termination proving of term rewrite systems or for completion-based theorem
provers. Well-known path orders are the lexicographic path order (LPO) \cite{LPO},
the recursive path order (RPO) \cite{RPO}, and the Knuth--Bendix order (KBO) \cite{KB70},
and all of these orders are presented in a standard textbook on term rewriting \cite[Chapter~5]{BN98}.

Whereas the mentioned path orders date back to the last century, the 
weighted path order (WPO) has only recently been presented \cite{WPO-PPDP,WPO}. It has two nice properties.
First, the search for suitable parameters is feasible and tools like NaTT and TTT2 implement it.
Second, WPO is quite powerful and versatile: in fact, KBO and LPO are just instances of WPO.
Moreover, with a slight extension of WPO (adding multiset-comparisons) also RPO is covered.

This AFP-entry provides a full formalization of WPO and also the connection to KBO, LPO, and RPO.
Here, for the existing formal version of KBO~\cite{KBO_form_paper,Knuth_Bendix_Order-AFP} it is just proven that WPO can simulate it by chosing suitable parameters, 
whereas LPO and RPO are defined from scratch and many properties  of LPO and RPO---such as strong normalization, closure under
contexts and substitutions, transitivity, etc.---are derived from the 
corresponding WPO properties.

Note that most of the WPO formalization is described in \cite{WPO_form_paper}.
The formal version deviates from the paper version
only by the additional possibility to perform multiset-comparisons instead
of lexicographic comparisons within WPO. 
The formal version of LPO and RPO extend their original definitions as well: the RPO definition is 
taken from \cite{RPO_def}, 
and LPO is defined as this extended RPO where always
lexicographic comparisons are performed when comparing lists of terms.
The formalization of multiset-comparisons (w.r.t.\ two orders) is described in more detail in \cite{RPO_def}.

\input{session}



\bibliographystyle{abbrv}
\bibliography{root}

\end{document}

