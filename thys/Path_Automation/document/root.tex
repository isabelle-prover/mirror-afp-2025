\documentclass[11pt,a4paper]{article}
\usepackage[utf8x]{inputenc}
\usepackage[LGR,T1]{fontenc}
\usepackage{isabelle,isabellesym}
\usepackage{amsfonts, amsmath, amssymb}
\usepackage{wasysym}
\usepackage[english]{babel}

% this should be the last package used
\usepackage{pdfsetup}
\usepackage{doi}
\usepackage[shortcuts]{extdash}

% urls in roman style, theory text in math-similar italics
\urlstyle{rm}
\isabellestyle{rm}


\begin{document}

\title{Path Equivalence and Automation\\ for Integration Contours}
\author{Manuel Eberl}
\maketitle

\begin{abstract}
\noindent 
In complex analysis, one often has to manipulate \emph{paths}, i.e.\ curves in the complex plane. 
This entry defines three useful relations on paths:
\begin{itemize}
\item an equivalence relation $\equiv_{\text{p}}$ that describes that two paths are the same up to 
  reparametrisation
\item a preorder $\leq_{\text{p}}$ that expresses the notion that one path is a subpath of another
\item an equivalence relation $\equiv_{\ocircle}$ that describes equivalence of closed paths up to 
  reparametrisation and ``shifting'' (e.g.\ if we have a rectangular path, it does not matter
  which corner we start in)
\end{itemize}
It also provides the \texttt{path} tactic, which proves or simplifies some common proof obligations
for composite paths. Namely:
\begin{itemize}
\item proving $\equiv_{\text{p}}$, $\leq_{\text{p}}$, $\equiv_{\ocircle}$
\item proving well-definedness of paths (\texttt{path}, \texttt{valid\_path})
\item determining the \emph{image} of a path (\texttt{path\_image})
\item showing that a path is not self-intersecting (\texttt{arc}, \texttt{simple\_path})
\item decomposing integrals on a composite path into the integrals on the constituent paths
\end{itemize}
\end{abstract}
\newpage

\tableofcontents

\newpage
\parindent 0pt\parskip 0.5ex

\input{session}

\raggedright

\bibliographystyle{abbrvurl}
\bibliography{root}

\end{document}

