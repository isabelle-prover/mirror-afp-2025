\documentclass[11pt,a4paper]{article}
\usepackage{isabelle,isabellesym}

% further packages required for unusual symbols (see also
% isabellesym.sty), use only when needed

\usepackage{latexsym}
\usepackage{amssymb}
\usepackage{amsmath}
  %for \<leadsto>, \<box>, \<diamond>, \<sqsupset>, \<mho>, \<Join>,
  %\<lhd>, \<lesssim>, \<greatersim>, \<lessapprox>, \<greaterapprox>,
  %\<triangleq>, \<yen>, \<lozenge>

%\usepackage{eurosym}
  %for \<euro>

%\usepackage[only,bigsqcap,bigparallel,fatsemi,interleave,sslash]{stmaryrd}
  %for \<Sqinter>, \<Parallel>, \<Zsemi>, \<Parallel>, \<sslash>

%\usepackage{eufrak}
  %for \<AA> ... \<ZZ>, \<aa> ... \<zz> (also included in amssymb)

%\usepackage{textcomp}
  %for \<onequarter>, \<onehalf>, \<threequarters>, \<degree>, \<cent>,
  %\<currency>

% this should be the last package used
\usepackage{pdfsetup}

% urls in roman style, theory text in math-similar italics
\urlstyle{rm}
\isabellestyle{it}

% for uniform font size
%\renewcommand{\isastyle}{\isastyleminor}


\begin{document}

\title{Providing restriction Spaces with an ultrametric Structure}
\author{Benoît Ballenghien \and Benjamin Puyobro \and Burkhart Wolff}
\maketitle

\abstract{
  We investigate the relationship between restriction spaces and classical metric
  structures by instantiating the former as ultrametric spaces.
  This is classically captured by defining the distance as
  $$\mathrm{dist} \ x \ y = \inf_{x \downarrow n \ = \ y \downarrow n} \left(\frac{1}{2}\right)^ n$$
  but we actually generalize this perspective by introducing a hierarchy of increasingly
  refined type classes to systematically relate ultrametric and restriction-based notions.
  This layered approach enables a precise comparison of structural and topological properties.
  In the end, our main result establishes that completeness in the sense of restriction spaces
  coincides with standard metric completeness, thus bridging the gap between
  \verb'Restriction_Spaces' and Banach's fixed-point theorem established in \verb'HOL-Analysis'.  
}

\tableofcontents

% sane default for proof documents
\parindent 0pt\parskip 0.5ex

% generated text of all theories
\input{session}

% optional bibliography
%\bibliographystyle{abbrv}
%\bibliography{root}

\end{document}

%%% Local Variables:
%%% mode: latex
%%% TeX-master: t
%%% End:
