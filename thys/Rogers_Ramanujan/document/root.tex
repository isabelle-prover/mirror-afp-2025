\documentclass[11pt,a4paper]{article}
\usepackage[utf8x]{inputenc}
\usepackage[LGR,T1]{fontenc}
\usepackage{isabelle,isabellesym}
\usepackage{amsfonts, amsmath, amssymb}
\usepackage[greek.polutoniko,greek,main=english]{babel}
\newcommand{\greekbous}{\textgreek{βοῦς}}
\newcommand{\greekstrofe}{\textgreek{στροφή}}
\newcommand{\greekdon}{\textgreek{-ηδόν}}
\DeclareMathOperator{\sech}{sech}
\newcommand{\oeiscite}[1]{\cite[\href{https://oeis.org/#1}{#1}]{oeis}}

% this should be the last package used
\usepackage{pdfsetup}
\usepackage[shortcuts]{extdash}

% urls in roman style, theory text in math-similar italics
\urlstyle{rm}
\isabellestyle{rm}


\begin{document}

\title{The Rogers--Ramanujan Identities}
\author{Manuel Eberl}
\maketitle

\begin{abstract}
This entry formalises the Rogers--Ramanujan Identities:
\begin{alignat*}{3}
  \sum_{k=-\infty}^\infty \frac{q^{k^2}}{\prod_{j=1}^k (1 - q^j)} &=
  \left(\,\prod_{n=0}^\infty (1 - q^{1+5n}) (1-q^{4+5n})\right)^{-1}\\
  \sum_{k=-\infty}^\infty \frac{q^{k^2+k}}{\prod_{j=1}^k (1 - q^j)} &=
  \left(\,\prod_{n=0}^\infty (1 - q^{2+5n}) (1-q^{3+5n})\right)^{-1}
\end{alignat*}

The formalisation follows the elegant proof given in Andrews and Eriksson 
\emph{Integer Partitions}, using the Jacobi triple product.
\end{abstract}

%\tableofcontents

\newpage
\parindent 0pt\parskip 0.5ex

\input{session}

\raggedright
\nocite{andrews2004}

\bibliographystyle{abbrv}
\bibliography{root}

\end{document}

