\documentclass[11pt,a4paper]{article}
\usepackage[T1]{fontenc}
\usepackage{isabelle,isabellesym}

\usepackage{amssymb}
\usepackage{authblk}

%\usepackage{eurosym}

%\usepackage[only,bigsqcap,bigparallel,fatsemi,interleave,sslash]{stmaryrd}

%\usepackage{eufrak}

%\usepackage{textcomp}

% this should be the last package used
\usepackage{pdfsetup}

% urls in roman style, theory text in math-similar italics
\urlstyle{rm}
\isabellestyle{it}

\newcommand\isafor{\textsf{Isa\kern-0.15exF\kern-0.15exo\kern-0.15exR}}
\newcommand\ceta{\textsf{C\kern-0.15exe\kern-0.45exT\kern-0.45exA}}

% for uniform font size
%\renewcommand{\isastyle}{\isastyleminor}


\begin{document}

\title{First-Order Rewriting}
\author{René Thiemann}
\author{Christian Sternagel}
\author{Christina Kirk (Kohl)}
\author{Martin Avanzini}
\author{Bertram Felgenhauer}
\author{Julian Nagele}
\author{Thomas Sternagel}
\author{Sarah Winkler}
\author{Akihisa Yamada}
\affil{University of Innsbruck, Austria}
\maketitle

\begin{abstract}
  This entry, derived from the \emph{Isabelle Formalization of Rewriting}
  (\isafor)~\cite{TS09}, provides a formalized foundation for first-order
  term rewriting. This serves as the basis for the certifier \ceta, which is
  generated from \isafor{} and verifies termination, confluence, and complexity
  proofs for term rewrite systems (TRSs).

  This formalization covers fundamental results for term rewriting, as presented
  in the foundational textbooks by Baader and Nipkow~\cite{BN98} and TeReSe~\cite{TeReSe}.
  These include:
  \begin{itemize}
    \item Various types of rewrite steps, such as root, ground, parallel, and multi-steps.
    \item Special cases of TRSs, such as linear and left-linear TRSs.
    \item A definition of critical pairs and key results, including the critical pair lemma.
    \item Orthogonality, notably that weak orthogonality implies confluence.
    \item Executable versions of relevant definitions, such as parallel and multi-step rewriting.
  \end{itemize}

\end{abstract}

\tableofcontents

\section{Introduction}

A TRS, as formalized here, is defined as a binary relation over first-order terms.
Given a TRS $\mathcal{R}$, a rule $(\ell, r) \in \mathcal{R}$ is typically
written as $\ell \to r$. The rewrite relation induced by $\mathcal{R}$,
denoted $\to_{\mathcal{R}}$, is defined as follows:
a term $s$ rewrites to a term $t$ using the TRS $\mathcal{R}$ (i.e,
$s \to_\mathcal{R} t$) if there are a context $C$, a substitution $\sigma$,
and a rule $(\ell, r) \in \mathcal{R}$ such that $s = C[\ell\sigma]$ and
$t = C[r\sigma]$.

The literature typically assumes two restrictions on the variables in a rule
$\ell \to r$ of a TRS: $\ell$ must not be a variable, and all variables in $r$
must appear in $\ell$. However, many results in term rewriting do not depend
on these conditions. In this entry, such constraints are enforced only where
necessary. A TRS that meets these criteria is called a \emph{well-formed} TRSs
(\isa{wf-trs}) in the formalization.

% sane default for proof documents
\parindent 0pt\parskip 0.5ex

% generated text of all theories
\input{session}

% optional bibliography
\bibliographystyle{abbrv}
\bibliography{root}

\end{document}

%%% Local Variables:
%%% mode: latex
%%% TeX-master: t
%%% End:
