\documentclass[11pt,a4paper]{article}
\usepackage[T1]{fontenc}
\usepackage{isabelle,isabellesym}

% further packages required for unusual symbols (see also
% isabellesym.sty), use only when needed

\usepackage{amssymb}
  %for \<leadsto>, \<box>, \<diamond>, \<sqsupset>, \<mho>, \<Join>,
  %\<lhd>, \<lesssim>, \<greatersim>, \<lessapprox>, \<greaterapprox>,
  %\<triangleq>, \<yen>, \<lozenge>

%\usepackage{eurosym}
  %for \<euro>

%\usepackage[only,bigsqcap,bigparallel,fatsemi,interleave,sslash]{stmaryrd}
  %for \<Sqinter>, \<Parallel>, \<Zsemi>, \<Parallel>, \<sslash>

%\usepackage{eufrak}
  %for \<AA> ... \<ZZ>, \<aa> ... \<zz> (also included in amssymb)

%\usepackage{textcomp}
  %for \<onequarter>, \<onehalf>, \<threequarters>, \<degree>, \<cent>,
  %\<currency>

% this should be the last package used
\usepackage{pdfsetup}

% urls in roman style, theory text in math-similar italics
\urlstyle{rm}
\isabellestyle{it}

% for uniform font size
%\renewcommand{\isastyle}{\isastyleminor}


\begin{document}

\title{Formal Proof of Dilworth's Theorem}
\author{Vivek Soorya Maadoori \and S. M. Meesum \and Shiv Pillai \and T. V. H. Prathamesh \and Aditya Swami}

\maketitle

\begin{abstract}
A \emph{chain} is defined as a totally ordered subset of a partially ordered set. A \emph{chain cover} refers to 
a collection of chains of a partially ordered set whose union equals the entire set. A \emph{chain 
decomposition} is a chain cover consisting of pairwise disjoint sets. An \emph{antichain} is a subset of
 elements of a partially ordered set in which no two elements are comparable.

In 1950, Dilworth proved that in any finite partially ordered set, the cardinality of a  largest antichain equals the cardinality of a smallest chain decomposition.\cite{dilworth1950}


In this paper, we formalise a proof of the theorem above, also known as \emph{Dilworth's theorem}, based 
on a proof by Perles (1963) \cite{perles1963proof}. Our formalisation draws on the formalisation of Dilworth's theorem 
for chain covers in Coq by Abhishek Kr. Singh \cite{singh2017fully}, and builds on the AFP entry containing formalisation of minimal and maximal elements in a set by Martin Desharnais \cite{Min_Max_Least_Greatest-AFP}. Our formalisation extends the prior work in Coq by including a formal proof of  Dilworth's theorem for chain decomposition.
\end{abstract}

\tableofcontents

% sane default for proof documents
\parindent 0pt\parskip 0.5ex

% generated text of all theories
\input{session}

\section*{Acknowledgement}
We would like thank Divakaran D. for valuable suggestions.

% optional bibliography
\bibliographystyle{abbrv}
\bibliography{root}

\end{document}

%%% Local Variables:
%%% mode: latex
%%% TeX-master: t
%%% End:
