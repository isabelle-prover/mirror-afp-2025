\documentclass[11pt,a4paper]{article}
\usepackage[T1]{fontenc}
\usepackage{isabelle,isabellesym}
\usepackage{geometry}
\usepackage{amsmath, amssymb}
\usepackage{wasysym}
\usepackage{stmaryrd}

% this should be the last package used
\usepackage{pdfsetup}

% urls in roman style, theory text in math-similar italics
\urlstyle{rm}
\isabellestyle{it}

% for uniform font size
%\renewcommand{\isastyle}{\isastyleminor}


\begin{document}

\title{Kraus Maps\thanks{Supported by the ERC consolidator grant CerQuS (819317), and the Estonian Cluster of Excellence ``Foundations of the Universe'' (TK202).}}
\author{Dominique Unruh}
\maketitle

\begin{abstract}
  We formalize Kraus maps \cite[Section 3]{kraus}, i.e., quantum channels of the form $\rho\mapsto\sum_x M_x\rho M_x^\dagger$ for suitable families of ``Kraus operators'' $M_x$.
  (In the finite-dimensional setting and the setting of separable Hilbert spaces,
  those are known to be equivalent to completely-positive maps, another common formalization of quantum channels.)
  Our results hold for arbitrary (i.e., not necessarily finite-dimensional or separable Hilbert spaces.

  Specifically, in theory \texttt{Kraus\_Families}, we formalize the type $(\alpha,\beta,\xi)\ \mathtt{kraus\_family}$ of families of Kraus operators $M_x$ (Kraus families for short),
  from trace-class operators on Hilbert space $\alpha$ to those on $\beta$, indexed by $x$ of type $\xi$.
  This induces both a Kraus map $\rho\mapsto\sum_x M_x\rho M_x^\dagger$, as well as a quantum measurement with outcomes of type $\xi$.

  We define and study various special Kraus families such as zero,
  identity, application of an operator, sum of two Kraus maps,
  sequential composition, infinite sum, random sampling, trace, tensor
  products, and complete measurements.

  Furthermore, since working with explicit Kraus families can be
  cumbersome when the specific family of operators is not relevant, in
  theory \texttt{Kraus\_Maps}, we define a Kraus map to be a function
  between two spaces that is of the form
  $\rho\mapsto\sum_x M_x\rho M_x^\dagger$ for some Kraus family, and
  restate our results in terms of such functions.  
\end{abstract}

\tableofcontents

% sane default for proof documents
\parindent 0pt\parskip 0.5ex

% generated text of all theories
\input{session}

% optional bibliography
\bibliographystyle{abbrv}
\bibliography{root}

\end{document}

%%% Local Variables:
%%% mode: latex
%%% TeX-master: t
%%% End:
