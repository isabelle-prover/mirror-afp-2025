\documentclass[11pt,a4paper]{article}
\usepackage[T1]{fontenc}
\usepackage{isabelle,isabellesym}

% this should be the last package used
\usepackage{pdfsetup}

% urls in roman style, theory text in math-similar italics
\urlstyle{rm}
\isabellestyle{literal}


\begin{document}

\title{Dyck Language}
\author{Tobias Nipkow and Moritz Roos}
\maketitle

\begin{abstract}
The Dyck language over a pair of brackets, e.g.\ ( and ), is the set
of balanced strings/words/lists of brackets. That is, the set of words
with the same number of ( and ), where every prefix of the word
contains no more ) than (. In general, a Dyck language is defined over
a whole set of matching pairs of brackets.
\end{abstract}

\tableofcontents

% include generated text of all theories
\input{session}

\bibliographystyle{abbrv}
\bibliography{root}

\end{document}
