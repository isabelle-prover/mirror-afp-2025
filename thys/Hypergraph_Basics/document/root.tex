\documentclass[11pt,a4paper]{article}
\usepackage[T1]{fontenc}
\usepackage{isabelle,isabellesym}

% this should be the last package used
\usepackage{pdfsetup}

% urls in roman style, theory text in math-similar italics
\urlstyle{rm}
\isabellestyle{it}


\begin{document}

\title{Hypergraph Basics}
\author{Chelsea Edmonds and Lawrence C. Paulson}
\maketitle

\begin{abstract}
    This entry is a simple extension of our previous entry for Combinatorial design theory \cite{Design_Theory-AFP}, which presents new and existing concepts using hypergraph language. Both designs and hypergraphs are types of incident set systems, hence have the same underlying foundation. However, they are often used in different contexts, and some definitions are as such unique. This library uses locales to rewrite equivalent definitions and build a basic hypergraph hierarchy with direct links to equivalent design theory concepts to avoid repetition, further demonstrating the power of the ``locale-centric'' approach. The library includes all standard definitions (order, degree etc.), as well as some extensions on hypergraph decompositions and spanning subhypergraphs.
\end{abstract}

\tableofcontents

% include generated text of all theories
\input{session}

\bibliographystyle{abbrv}
\bibliography{root}

\end{document}
