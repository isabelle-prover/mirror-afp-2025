\documentclass[11pt,a4paper]{article}
\usepackage[T1]{fontenc}
\usepackage{isabelle,isabellesym}
\usepackage{latexsym}
\usepackage{amsmath}
\usepackage{amssymb}
\usepackage[english]{babel}
\usepackage{wasysym}
\usepackage{eufrak}
\usepackage{textcomp}

% this should be the last package used
\usepackage{pdfsetup}

% urls in roman style, theory text in math-similar italics
\urlstyle{rm}
\isabellestyle{it}


\begin{document}

\title{The Kolmogorov-Chentsov Theorem}
\author{Christian Pardillo Laursen and Simon Foster\\[.5ex] University of York, UK \\[2ex] \texttt{\small $\{$christian.laursen,simon.foster$\}$@york.ac.uk}}
\maketitle

\begin{abstract}
Continuous-time stochastic processes often carry the condition of having
almost-surely continuous paths. If some process \(X\) satisfies certain
bounds on its expectation, then the Kolmogorov-Chentsov theorem lets us
construct a modification of \(X\), i.e. a process \(X'\) such that
\(\forall t. X_t = X'_t\) almost surely, that has H{\"o}lder continuous
paths.

In this work, we mechanise the Kolmogorov-Chentsov theorem. To get there, we
develop a theory of stochastic processes, together with H{\"o}lder continuity,
convergence in measure, and arbitrary intervals of dyadic rationals.

With this, we pave the way towards a construction of Brownian motion. The
work is based on the exposition in Achim Klenke's probability theory text~\cite{klenke2020}.
\end{abstract}

\tableofcontents

% sane default for proof documents
\parindent 0pt\parskip 0.5ex

% include generated text of all theories
\input{session}

\vspace{4ex}

\bibliographystyle{abbrv}
\bibliography{root}

\end{document}
