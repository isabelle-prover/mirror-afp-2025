\documentclass[11pt,a4paper]{article}
\usepackage{isabelle,isabellesym}
\usepackage{amsfonts, amsmath, amssymb}

% this should be the last package used
\usepackage{pdfsetup}
\usepackage[shortcuts]{extdash}

% urls in roman style, theory text in math-similar italics
\urlstyle{rm}
\isabellestyle{rm}


\begin{document}

\title{The Sum-of-Squares Function and Jacobi's Two-Square Theorem}
\author{Manuel Eberl}
\maketitle

\begin{abstract}
This entry defines the \emph{sum-of-squares function} $r_k(n)$, which counts the number of
ways to write a natural number $n$ as a sum of $k$ squares of integers. Signs and permutations of 
these integers are taken into account, such that e.g.\ $1^2+2^2$, $2^2+1^2$, and $(-1)^2+2^2$
are all different decompositions of $5$.

Using this, I then formalise the main result: Jacobi's two-square theorem, which states that for $n > 0$
we have $r_2(n) = 4(d_1(3) - d_3(n)),$ where $d_i(n)$ denotes the number of divisors $m$ of $n$
such that $m = i\ (\text{mod}\ 4)$.

Corollaries include the identities $r_2(2n) = r_2(n)$ and $r_2(p^2n) = r_2(n)$ if
$p = 3\ (\text{mod}\ 4)$ and the well-known theorem that $r_2(n) = 0$ iff $n$ has a prime
factor $p$ of odd multiplicity with $p = 3\ (\text{mod}\ 4)$.
\end{abstract}

\tableofcontents

\newpage
\parindent 0pt\parskip 0.5ex

\input{session}

\raggedright
\nocite{grosswald2012}
\bibliographystyle{abbrv}
\bibliography{root}

\end{document}

