\documentclass[11pt,a4paper]{article}
\usepackage[T1]{fontenc}
\usepackage{isabelle,isabellesym}

% further packages required for unusual symbols (see also
% isabellesym.sty), use only when needed

\usepackage{amssymb}
  %for \<leadsto>, \<box>, \<diamond>, \<sqsupset>, \<mho>, \<Join>,
  %\<lhd>, \<lesssim>, \<greatersim>, \<lessapprox>, \<greaterapprox>,
  %\<triangleq>, \<yen>, \<lozenge>
\usepackage{amsmath}

%\usepackage{eurosym}
  %for \<euro>

\usepackage[only,bigsqcap]{stmaryrd}
%for \<Sqinter>

%\usepackage{eufrak}
  %for \<AA> ... \<ZZ>, \<aa> ... \<zz> (also included in amssymb)

\usepackage{mathrsfs}
% for mathscr

%\usepackage{textcomp}
  %for \<onequarter>, \<onehalf>, \<threequarters>, \<degree>, \<cent>,
  %\<currency>

% this should be the last package used
\usepackage{pdfsetup}

% urls in roman style, theory text in math-similar italics
\urlstyle{rm}
\isabellestyle{it}


% for uniform font size
%\renewcommand{\isastyle}{\isastyleminor}


\begin{document}

\title{S-Finite Measure Monad on Quasi-Borel Spaces}
\author{Michikazu Hirata, Yasuhiko Minamide}
\maketitle
\begin{abstract}
  The s-finite measure monad on quasi-Borel spaces provides
  a suitable denotational model for higher-order probabilistic programs
  with conditioning.
  This entry is a formalization of the s-finite measure monad and related notions,
  including s-finite measures, s-finite kernels, and a proof automation for quasi-Borel spaces which is an
  extension of our previous entry \textit{quasi-Borel spaces}.
  We also implement several examples of probabilistic programs in previous works and prove their property.

  This work is a part of the work by Hirata, Minamide, and Sato,
  \textit{Semantic Foundations of Higher-Order Probabilistic Programs in Isabelle/HOL}
   which will be presented at the 14th Conference on Interactive Theorem Proving (ITP2023).
\end{abstract}

\tableofcontents

% sane default for proof documents
\parindent 0pt\parskip 0.5ex

% generated text of all theories
\input{session}

% optional bibliography
\bibliographystyle{abbrv}
\bibliography{root}

\end{document}

%%% Local Variables:
%%% mode: latex
%%% TeX-master: t
%%% End:
