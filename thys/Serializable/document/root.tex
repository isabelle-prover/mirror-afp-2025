\documentclass[11pt,a4paper]{article}
\usepackage[T1]{fontenc}
\usepackage{isabelle,isabellesym}

% further packages required for unusual symbols (see also
% isabellesym.sty), use only when needed

%\usepackage{amssymb}
  %for \<leadsto>, \<box>, \<diamond>, \<sqsupset>, \<mho>, \<Join>,
  %\<lhd>, \<lesssim>, \<greatersim>, \<lessapprox>, \<greaterapprox>,
  %\<triangleq>, \<yen>, \<lozenge>

%\usepackage{eurosym}
  %for \<euro>

%\usepackage[only,bigsqcap,bigparallel,fatsemi,interleave,sslash]{stmaryrd}
  %for \<Sqinter>, \<Parallel>, \<Zsemi>, \<Parallel>, \<sslash>

%\usepackage{eufrak}
  %for \<AA> ... \<ZZ>, \<aa> ... \<zz> (also included in amssymb)

%\usepackage{textcomp}
  %for \<onequarter>, \<onehalf>, \<threequarters>, \<degree>, \<cent>,
  %\<currency>

% this should be the last package used
\usepackage{pdfsetup}

% urls in roman style, theory text in math-similar italics
\urlstyle{rm}
\isabellestyle{it}

% for uniform font size
%\renewcommand{\isastyle}{\isastyleminor}


\begin{document}

\title{Formalization of (Conflict-)Serializability and Strict~Two-Phase~Locking}
\author{Dmitriy Traytel}
\maketitle

\begin{abstract} Concurrency control is an essential component of any transactional database management system, which is responsible for providing isolation (the ``I'' in ACID) to
transactions. Formally, concurrency control aims to achieve serializability: a way to rearrange the actions of concurrently executing transactions that eliminates concurrency while leaves the
database modifications unchanged. In this small entry, we define serializability, a syntactic over-approximation called conflict-serializability, and characterize schedules generated by the
frequently used concurrency control mechanism of strict two-phase locking (S2PL). We also prove two inclusions: S2PL implies conflict-serializabili\-ty, which in turn implies serializability.
The formalization is based on standard material from an advanced database systems course \cite[Chapter 17]{ramakrishnan2003database}. \end{abstract}

%\tableofcontents

% sane default for proof documents
\parindent 0pt\parskip 0.5ex

% generated text of all theories
\input{session}

% optional bibliography
\bibliographystyle{abbrv}
\bibliography{root}

\end{document}

%%% Local Variables:
%%% mode: latex
%%% TeX-master: t
%%% End:
