\documentclass[11pt,a4paper]{article}
\usepackage[T1]{fontenc}
\usepackage{isabelle,isabellesym}

% further packages required for unusual symbols (see also
% isabellesym.sty), use only when needed

\usepackage{amssymb}
  %for \<leadsto>, \<box>, \<diamond>, \<sqsupset>, \<mho>, \<Join>,
  %\<lhd>, \<lesssim>, \<greatersim>, \<lessapprox>, \<greaterapprox>,
  %\<triangleq>, \<yen>, \<lozenge>

%\usepackage{eurosym}
  %for \<euro>

%\usepackage[only,bigsqcap]{stmaryrd}
  %for \<Sqinter>

%\usepackage{eufrak}
  %for \<AA> ... \<ZZ>, \<aa> ... \<zz> (also included in amssymb)

%\usepackage{textcomp}
  %for \<onequarter>, \<onehalf>, \<threequarters>, \<degree>, \<cent>,
  %\<currency>

% this should be the last package used
\usepackage{pdfsetup}

% urls in roman style, theory text in math-similar italics
\urlstyle{rm}
\isabellestyle{it}

% for uniform font size
%\renewcommand{\isastyle}{\isastyleminor}


\begin{document}

\title{SCL(FOL) Can Simulate Ground Nonredundant Ordered Resolution}
\author{Martin Desharnais}
\maketitle

\begin{abstract}
  SCL(FOL) (i.e., Simple Clause Learning for First-Order Logic without equality) is known to be able to simulate the derivation of nonredundant clauses by the ground ordered resolution calculus \cite{Bromberger_CADE2023}.
  Due to the space constraints of a 16-pages paper, the published proof is monolithic and hard to comprehend.
  In this work, we reuse the existing strategy for ground ordered resolution and present a new, simpler strategy for SCL(FOL).
  We prove a stronger bisimulation theorem between these two strategies (i.e., they both simulate each other).
  Our proof is modular: it consists of ten refinement steps focusing on different aspects of the two strategies.
\end{abstract}

\tableofcontents

% sane default for proof documents
\parindent 0pt\parskip 0.5ex

% generated text of all theories
\input{session}

% optional bibliography
\bibliographystyle{abbrv}
\bibliography{root}

\end{document}

%%% Local Variables:
%%% mode: latex
%%% TeX-master: t
%%% End:
