\documentclass[11pt,a4paper]{article}
\usepackage[utf8x]{inputenc}
\usepackage[LGR,T1]{fontenc}
\usepackage{isabelle,isabellesym}
\usepackage{amsfonts, amsmath, amssymb}
\usepackage[greek.polutoniko,greek,main=english]{babel}
\newcommand{\greekbous}{\textgreek{βοῦς}}
\newcommand{\greekstrofe}{\textgreek{στροφή}}
\newcommand{\greekdon}{\textgreek{-ηδόν}}
\DeclareMathOperator{\sech}{sech}
\newcommand{\oeiscite}[1]{\cite[\href{https://oeis.org/#1}{#1}]{oeis}}

% this should be the last package used
\usepackage{pdfsetup}
\usepackage[shortcuts]{extdash}

% urls in roman style, theory text in math-similar italics
\urlstyle{rm}
\isabellestyle{rm}


\begin{document}

\title{Theta Functions}
\author{Manuel Eberl}
\maketitle

\begin{abstract}
This entry defines the Ramanujan theta function
\[f(a,b) = \sum_{n=-\infty}^\infty a^{\frac{n(n+1)}{2}} b^{\frac{n(n-1)}{2}}\]
and derives from it the more commonly known Jacobi theta function on the unit disc
\[\vartheta_{00}(w,q) = \sum_{n=-\infty}^\infty w^{2n} q^{n^2},\ \]
its version in the complex plane
\[\vartheta_{00}(z;\tau) = \sum_{n=-\infty}^\infty \exp(i\pi (2nz + n^2\tau))\]
as well as its half-period variants $\vartheta_{01}$, $\vartheta_{10}$, and $\vartheta_{11}$.

The most notable single result in this work is the proof of Jacobi's triple product
\[\prod_{n=1}^\infty (1-q^{2m})(1+q^{2m-1}w^2)(1+q^{2m-1}w^{-2}) = 
  \sum_{k=-\infty}^\infty q^{k^2}w^{2k}\]
and its corollary, Euler's famous pentagonal number theorem:
\[\prod_{n=1}^\infty (1-q^n) = \sum_{k=-\infty}^\infty (-1)^k q^{k(3k-1)/2}\]
\end{abstract}

\newpage

\tableofcontents

\newpage
\parindent 0pt\parskip 0.5ex

\section{Auxiliary material}

\input{session}

\raggedright
\nocite{brent2020}
\nocite{borwein1987}

\bibliographystyle{abbrv}
\bibliography{root}

\end{document}

