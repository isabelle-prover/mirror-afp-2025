\documentclass[11pt,a4paper]{article}
\usepackage[T1]{fontenc}
\usepackage{isabelle,isabellesym}

\usepackage{amsmath}


% further packages required for unusual symbols (see also
% isabellesym.sty), use only when needed

\usepackage{amssymb}
  %for \<leadsto>, \<box>, \<diamond>, \<sqsupset>, \<mho>, \<Join>,
  %\<lhd>, \<lesssim>, \<greatersim>, \<lessapprox>, \<greaterapprox>,
  %\<triangleq>, \<yen>, \<lozenge>

%\usepackage{eurosym}
  %for \<euro>

%\usepackage[only,bigsqcap,bigparallel,fatsemi,interleave,sslash]{stmaryrd}
  %for \<Sqinter>, \<Parallel>, \<Zsemi>, \<Parallel>, \<sslash>

%\usepackage{eufrak}
  %for \<AA> ... \<ZZ>, \<aa> ... \<zz> (also included in amssymb)

%\usepackage{textcomp}
  %for \<onequarter>, \<onehalf>, \<threequarters>, \<degree>, \<cent>,
  %\<currency>

% this should be the last package used
\usepackage{pdfsetup}

% urls in roman style, theory text in math-similar italics
\urlstyle{rm}
\isabellestyle{it}

% for uniform font size
%\renewcommand{\isastyle}{\isastyleminor}


\begin{document}

\title{Definition and Elementary Properties of Ultrametric Spaces}
\author{
Benoît Ballenghien\footnote{Université Paris-Saclay, CNRS, ENS Paris-Saclay, LMF\\\url{https://orcid.org/0009-0000-4941-187X}} \and
Benjamin Puyobro\footnote{Université Paris-Saclay, IRT SystemX, CNRS, ENS Paris-Saclay, LMF\\\url{https://orcid.org/0009-0006-0294-9043}\\
This work has been supported by the French government under the ``France 2030''
program, as part of the SystemX Technological Research Institute within the CVH project.} \and
Burkhart Wolff\footnote{Université Paris-Saclay, CNRS, ENS Paris-Saclay, LMF\\\url{https://orcid.org/0000-0002-9648-7663}}}

\maketitle

\abstract{
An ultrametric space is a metric space in which the triangle inequality is strengthened
by using the maximum instead of the sum. More formally, such a space is equipped with
a real-valued application $dist$, called distance, verifying the four following conditions.
\begin{align*}
    & dist \ x \ y \ge 0\\
    & dist \ x \ y = dist \ y \ x\\
    & dist \ x \ y = 0 \longleftrightarrow x = y\\
    & dist \ x \ z \le max \ (dist \ x \ y) \ (dist \ y \ z)
\end{align*}
In this entry, we present an elementary formalization of these spaces relying on axiomatic type classes.
The connection with standard metric spaces is obtained through a subclass relationship,
and fundamental properties of ultrametric spaces are formally established.
}

\newpage

\tableofcontents


% sane default for proof documents
\parindent 0pt\parskip 0.5ex

% generated text of all theories
\input{session}

% optional bibliography
%\bibliographystyle{abbrv}
%\bibliography{root}

\end{document}

%%% Local Variables:
%%% mode: latex
%%% TeX-master: t
%%% End:
