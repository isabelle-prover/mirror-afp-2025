\documentclass[11pt,a4paper]{article}
\usepackage[T1]{fontenc}
\usepackage{amssymb}
\usepackage{isabelle,isabellesym}
\usepackage[english]{babel}  % for guillemots

% this should be the last package used
\usepackage{pdfsetup}

% urls in roman style, theory text in math-similar italics
\urlstyle{rm}
\isabellestyle{it}

\begin{document}

\title{A Generalization of the Cauchy--Davenport Theorem}
\author{Mantas Bak\v{s}ys \\
University of Cambridge\\
\texttt{mb2412@cam.ac.uk}}

\maketitle
                                                             
\begin{abstract}
The Cauchy--Davenport theorem is a fundamental result in additive combinatorics.
It was originally independently discovered by Cauchy \cite{cauchy1812recherches} and Davenport \cite{davenport} and has been formalized in the AFP entry \cite{Kneser_Cauchy_Davenport-AFP} as a corollary of Kneser's theorem.
More recently, many generalizations of this theorem have been found. In this entry, we formalise a generalization due to DeVos \cite{DeVos2016OnAG}, which proves the theorem in a non-abelian setting.
\end{abstract}
\newpage
\tableofcontents

\newpage

% include generated text of all theories
\input{session}


\bibliographystyle{abbrv}
\bibliography{root}


\end{document}
