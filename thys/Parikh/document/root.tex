\documentclass[11pt,a4paper]{article}
\usepackage[T1]{fontenc}
\usepackage{isabelle,isabellesym}

% this should be the last package used
\usepackage{pdfsetup}

% urls in roman style, theory text in math-similar italics
\urlstyle{rm}
\isabellestyle{literal}


\begin{document}

\title{Parikh's theorem}
\author{Fabian Lehr}
\maketitle

\begin{abstract}
  In formal language theory, the \emph{Parikh image} of a language $L$
  is the set of multisets of the words in $L$: the order of letters
  becomes irrelevant, only the number of occurrences is relevant.
  Parikh's Theorem states that the Parikh image of a context-free
  language is the same as the Parikh image of some regular language.
  This formalization closely follows Pilling's proof \cite{Pilling}:
  It describes a context-free language as a minimal solution to a
  system of equations induced by a context free grammar for this
  language. Then it is shown that there exists a minimal solution
  to this system which is regular, such that the regular solution and the
  context-free language have the same Parikh image.
\end{abstract}

\tableofcontents

% include generated text of all theories
\input{session}

\bibliographystyle{abbrv}
\bibliography{root}

\end{document}
