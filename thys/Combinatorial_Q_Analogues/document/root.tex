\documentclass[11pt,a4paper]{article}
\usepackage[utf8x]{inputenc}
\usepackage[LGR,T1]{fontenc}
\usepackage{isabelle,isabellesym}
\usepackage{amsfonts, amsmath, amssymb}
\usepackage[greek.polutoniko,greek,main=english]{babel}
\newcommand{\greekbous}{\textgreek{βοῦς}}
\newcommand{\greekstrofe}{\textgreek{στροφή}}
\newcommand{\greekdon}{\textgreek{-ηδόν}}
\DeclareMathOperator{\sech}{sech}
\newcommand{\oeiscite}[1]{\cite[\href{https://oeis.org/#1}{#1}]{oeis}}

% this should be the last package used
\usepackage{pdfsetup}
\usepackage[shortcuts]{extdash}

% urls in roman style, theory text in math-similar italics
\urlstyle{rm}
\isabellestyle{rm}


\begin{document}

\title{Combinatorial $q$-Analogues}
\author{Manuel Eberl}
\maketitle

\begin{abstract}
This entry defines the $q$-analogues of various combinatorial symbols, namely:
\begin{itemize}
  \item The $q$-bracket $[n]_q = \frac{1-q^n}{1-q}$ for $n\in\mathbb{Z}$
  \item The $q$-factorial $[n]_q! = [1]_q [2]_q \cdots [n]_q$ for $n\in\mathbb{Z}$
  \item The $q$-binomial coefficients $\binom{n}{k}_{\!q} = \frac{[n]_q!}{[k]_q!\,[n-k]_q!}$ for $n,k\in\mathbb{N}$ (also known as
        Gaussian binomial coefficients or Gaussian polynomials)
  \item The infinite $q$-Pochhammer symbol $(a; q)_\infty = \prod_{n=0}^\infty\, (1 - aq^n)$
  \item Euler's $\phi$ function $\phi(q) = (q; q)_\infty$
  \item The finite $q$-Pochhammer symbol $(a; q)_n = (a; q)_\infty / (aq^n; q)_\infty$ for $n\in\mathbb{Z}$
\end{itemize}
Proofs for many basic properties are provided, notably for the $q$-binomial theorem:
\[(-a; q)_n = \prod_{k=0}^{n-1} (1 + aq^n) = \sum_{k=0}^n \binom{n}{k}_{\!\!q} a^k q^{k(k-1)/2} \]
Additionally, two identities of Euler are formalised that give power series expansions for
$(a; q)_\infty$ and $1/(a; q)_\infty$ in powers of $a$:
\begin{alignat*}{3}
(a; q)_\infty &{}= \prod_{k=0}^\infty (1 - aq^k) &{}= 
      \sum_{n=0}^\infty \frac{(-a)^n q^{n(n-1)/2}}{(1-q) \cdots (1-q^n)}\\
\frac{1}{(a; q)_\infty} &{}= \prod_{k=0}^\infty \frac{1}{1 - aq^k} &{}= 
      \sum_{n=0}^\infty \frac{a^n}{(1-q)\cdots(1-q^n)}
\end{alignat*}

\end{abstract}

\newpage
\tableofcontents

\newpage
\parindent 0pt\parskip 0.5ex

\section{Auxiliary material}

\input{session}

\raggedright
\nocite{andrews1999}

\bibliographystyle{abbrv}
\bibliography{root}

\end{document}

