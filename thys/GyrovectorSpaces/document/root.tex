\documentclass[11pt,a4paper]{article}
\usepackage[T1]{fontenc}
\usepackage{isabelle,isabellesym}
\usepackage{eufrak}
\usepackage{amsmath, amssymb}

% this should be the last package used
\usepackage{pdfsetup}

% urls in roman style, theory text in math-similar italics
\urlstyle{rm}
\isabellestyle{it}
\usepackage{hyperref}
\usepackage{url}

\begin{document}

\title{Formalization of Gyrovector Spaces as Models of Hyperbolic Geometry and Special Relativity}
\author{Jelena Markovi\' c and Filip Mari\' c}
\maketitle

\begin{abstract}
  In this paper, we present an Isabelle/HOL formalization of
  noncommu\-ta\-ti\-ve and nonassociative algebraic structures known
  as \emph{gyrogroups} and \emph{gyrovector spaces}. These concepts
  were introduced by Abraham A. Ungar \cite{ungar-analytic} and have deep connections to
  hyperbolic geometry and special relativity. Gyrovector spaces can be
  used to define models of hyperbolic geometry. Unlike other models,
  gyrovector spaces offer the advantage that all definitions exhibit
  remarkable syntactical similarities to standard Euclidean and
  Cartesian geometry (e.g., points on the line between $a$ and $b$
  satisfy the parametric equation
  $ a \oplus t \otimes(\ominus a \oplus b)$, for $t \in \mathbb{R}$,
  while the hyperbolic Pythagorean theorem is expressed as
  $a^2\oplus b^2 = c^2$, where $\otimes$, $\oplus$, and $\ominus$
  represent gyro operations).

  We begin by formally defining gyrogroups and gyrovector spaces and
  proving their numerous properties. Next, we formalize M\"obius and
  Einstein models of these abstract structures (formulated in the
  two-dimensional, complex plane), and then demonstrate that these are
  equivalent to the Poincar\'e and Klein-Beltrami models, satisfying
  Tarski's geometry axioms for hyperbolic geometry.
\end{abstract}

\tableofcontents

% include generated text of all theories
\input{session}

\bibliographystyle{abbrv}
\bibliography{root}

\end{document}
