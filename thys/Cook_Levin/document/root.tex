\documentclass[10pt,a4paper]{report}
\usepackage[T1]{fontenc}
\usepackage{isabelle,isabellesym}

\usepackage{geometry}
\geometry{a4paper, portrait}
\geometry{left=2.5cm, right=2.5cm, top=2.5cm, bottom=2.5cm}

% further packages required for unusual symbols (see also
% isabellesym.sty), use only when needed

\usepackage{amssymb}  % for \mathbb
\usepackage{amsmath}  % for \tag

% this should be the last package used
\usepackage{pdfsetup}

% urls in roman style, theory text in math-similar italics
\urlstyle{rm}
\isabellestyle{it}

% for uniform font size
%\renewcommand{\isastyle}{\isastyleminor}

\renewcommand{\phi}{\varphi}
\renewcommand{\rho}{\varrho}
\newcommand{\nat}{\mathbb{N}}
\newcommand{\NP}{\mathcal{NP}}
\newcommand{\SAT}{\texttt{SAT}}
\newcommand{\bbOI}{\{\isasymbbbO,\isasymbbbI\}}
\newcommand{\prev}{\mathit{prev}}
\newcommand{\inputpos}{\mathit{inputpos}}

\begin{document}

\title{The Cook-Levin theorem}
\author{Frank J. Balbach}
\maketitle

\begin{abstract}
The Cook-Levin theorem states that deciding the satisfiability of Boolean
formulas in conjunctive normal form is $\NP$-complete.  This entry
formalizes a proof of this theorem based on the textbook \emph{Computational
Complexity:\ A Modern Approach} by Arora and Barak.  It contains definitions
of deterministic multi-tape Turing machines, the complexity classes
$\mathcal{P}$ and $\NP$, polynomial-time many-one reduction, and the decision
problem \SAT.  For the $\NP$-hardness of \SAT, the proof first shows that every
polynomial-time computation can be performed by a two-tape oblivious Turing
machine. An $\NP$ problem can then be reduced to \SAT{} by a polynomial-time
Turing machine that encodes computations of the problem's oblivious two-tape
verifier Turing machine as formulas in conjunctive normal form.
\end{abstract}

\tableofcontents

% sane default for proof documents
\parindent 0pt\parskip 0.5ex

% generated text of all theories
\input{session}

% optional bibliography
\bibliographystyle{plain}
\bibliography{root}

\end{document}
