\documentclass[11pt,a4paper]{article}
\usepackage[T1]{fontenc}
\usepackage{isabelle,isabellesym}
\usepackage{amsfonts, amsmath, amssymb}

\usepackage{tikz}
\usepackage{pgfplots}

% this should be the last package used
\usepackage{pdfsetup}
\usepackage[shortcuts]{extdash}

% urls in roman style, theory text in math-similar italics
\urlstyle{rm}
\isabellestyle{rm}


\begin{document}

\title{Chebyshev Polynomials}
\author{Manuel Eberl}
\maketitle

\begin{abstract}
The multiple-angle formulas for $\cos$ and $\sin$ state that for any natural number $n$, 
the values of $\cos nx$ and $\sin nx$ can be expressed in terms of $\cos x$ and $\sin x$.
To be more precise, there are polynomials $T_n$ and $U_n$ such that $\cos nx = T_n(\cos x)$
and $\sin nx = U_n(\cos x)\sin x$. These are called the \emph{Chebyshev polynomials of the
first and second kind}, respectively.

This entry contains a definition of these two familes of polynomials in Isabelle/HOL
along with some of their most important properties. In particular, it is shown that $T_n$ and $U_n$
are \emph{orthogonal} families of polynomials.

Moreover, we show the well-known result that for any monic polynomial $p$ of degree $n > 0$,
it holds that $\sup_{x\in[-1,1]} |p(x)| \geq 2^{n-1}$, and that this inequality is sharp since
equality holds with $p = 2^{1-n} T_n$. This has important consequences in the theory of
function interpolation, since it implies that the roots of $T_n$ (also colled the
\emph{Chebyshev nodes}) are exceptionally well-suited as interpolation nodes.
\end{abstract}


\newpage
\tableofcontents

\newpage
\parindent 0pt\parskip 0.5ex

\input{session}

\nocite{mason2002chebyshev}
\raggedright
\bibliographystyle{abbrv}
\bibliography{root}

\end{document}

